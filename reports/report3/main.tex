\documentclass[a4paper,12pt]{article}
\usepackage[utf8]{inputenc}
\usepackage[polish]{babel}
\usepackage{graphicx}
\usepackage{amsmath}
\usepackage{hyperref}
\usepackage{geometry}
\usepackage{fancyhdr}
\usepackage{titling}
\usepackage{natbib}
\usepackage{listings}
\usepackage{amsmath}
\usepackage{graphicx}
\usepackage{hyperref}
\usepackage{listings}
\usepackage{color}
\usepackage{geometry}
\usepackage{float} % pakiet do bardziej precyzyjnego 
\geometry{margin=1in}
\usepackage{amsmath}
\usepackage{hyperref}
\definecolor{codegreen}{rgb}{0,0.6,0}
\definecolor{codegray}{rgb}{0.5,0.5,0.5}
\definecolor{codepurple}{rgb}{0.58,0,0.82}
\definecolor{backcolour}{rgb}{0.95,0.95,0.92}
\usepackage[T1]{fontenc}

\title{Modelowanie i Symulacja Systemów - Algorytm Stadny}
\author{Dariusz Królicki, Adam Staniszewski}
\date{\today}

\begin{document}

\maketitle

\section{Wprowadzenie}
Algorytm stadny (flocking algorithm) został po raz pierwszy zaproponowany przez Craiga Reynoldsa w 1986 roku. Reynolds opracował ten algorytm, aby symulować realistyczne zachowanie stad zwierząt, takich jak ptaki lub ryby, w środowisku komputerowym. Oparty jest na trzech podstawowych zasadach: separacji, kohezji i wyrównaniu. Separacja zapobiega kolizjom między osobnikami, kohezja utrzymuje grupę razem, a wyrównanie pozwala poszczególnym osobnikom na synchronizację swojego ruchu z sąsiadami.\\

Algorytm stadny znalazł szerokie zastosowanie nie tylko w grafice komputerowej i animacji, ale także w robotyce, gdzie wykorzystywany jest do sterowania grupami dronów lub robotów mobilnych. Ponadto, jego zasady są stosowane w modelowaniu ruchu tłumu oraz analizie zachowań społecznych różnych gatunków zwierząt. Dzięki swojej prostocie i efektywności, algorytm stadny stał się podstawowym narzędziem w badaniach nad systemami wieloagentowymi i złożonymi systemami adaptacyjnymi.
\section{Podstawy teoretyczne}
Jak wspomniano, algorytm opiera się na trzech zasadach: separacji, kohezji oraz wyrównaniu. Poniżej przedstawiono wzory umożliwiające opis tych zasad.
\subsection{Separacja}
Celem separacji jest unikanie kolizji między agentami. Każdy agent oblicza siłę separacji, która jest wektorem przeciwnym do wektora pozycji najbliższych sąsiadów. Matematycznie można to wyrazić jako:
\begin{equation}
    \mathbf{S}_i = - \sum_{j \in N_i} (\mathbf{p}_j - \mathbf{p}_i)
\end{equation}
gdzie:
\begin{itemize}
    \item $S_{i}$ to wektor separacji dla agenta \textit{i}
    \item $N_{i}$ to zbiór sąsiadów agenta \textit{i}
    \item $p_{j}$ to pozycja sąsiada \textit{j}
    \item $p_{i}$ to pozycja agenta \textit{i}
\end{itemize}

\subsection{Kohezja}
Celem kohezji jest utrzymanie grupy razem poprzez przesuwanie się agentów w kierunku środka masy ich sąsiadów. Środek masy można obliczyć jako średnią pozycję sąsiadów, a siłę kohezji jako wektor prowadzący do tego środka:
\begin{equation}
    \mathbf{C}_i = \left( \frac{1}{|N_i|} \sum_{j \in N_i} \mathbf{p}_j \right) - \mathbf{p}_i
\end{equation}
gdzie:
\begin{itemize}
    \item $C_{i}$ to wektor kohezji dla agenta \textit{i}
    \item $|N_{i}|$ to liczba sąsiadów agenta \textit{i}
    \item $p_{j}$ to pozycja sąsiada \textit{j}
    \item $p_{i}$ to pozycja agenta \textit{i}
\end{itemize}

\subsection{Wyrównanie}
Celem wyrównania jest synchronizacja kierunku i prędkości ruchu agentów z ich sąsiadami. Każdy agent oblicza średni wektor prędkości sąsiadów i dostosowuje swoją prędkość w tym kierunku:
\begin{equation}
    \mathbf{A}_i = \frac{1}{|N_i|} \sum_{j \in N_i} \mathbf{v}_j - \mathbf{v}_i
\end{equation}
gdzie:
\begin{itemize}
    \item $A_{i}$ to wektor wyrównania dla agenta \textit{i}
    \item $v_{j}$ to prędkość sąsiada \textit{j}
    \item $v_{i}$ to prędkość agenta \textit{i}
\end{itemize}

\subsection{Wypadkowa}
Powyższe trzy współczynniki przyczyniają się do stworzenia wektora siły całkowitej działającej na agenta. Są one przemnożone przez odpowiednie wagi, pozwalając na optymalizację algorytmu:
\begin{equation}
    \mathbf{F}_i = w_s \mathbf{S}_i + w_c \mathbf{C}_i + w_a \mathbf{A}_i
\end{equation}
gdzie $F_{i}$ to całkowita siła działająca na agenta \textit{i}, a $w_{s}$, $w_{c}$, $w_{a}$ to wagi odpowiednio dla separacji, kohezji oraz wyrównania. \\
Po obliczeniu całkowitej siły, aktualizowane są pozycje i prędkości agentów zgodnie z zasadami dynamiki:
\begin{equation}
    \mathbf{v}_i(t+1) = \mathbf{v}_i(t) + \mathbf{F}_i
\end{equation}
\begin{equation}
    \mathbf{p}_i(t+1) = \mathbf{p}_i(t) + \mathbf{v}_i(t+1)
\end{equation}
gdzie:
\begin{itemize}
    \item $v_{i}(t+1)$ to prędkość agenta \textit{i} w chwili \textit{t+1}
    \item $p_{i}(t+1)$ to pozycja agenta \textit{i} w chwili \textit{t+1}
    \item $v_{i}$ to prędkość agenta \textit{i} w chwili \textit{t}
    \item $p_{i}$ to pozycja agenta \textit{i} w chwili \textit{t}
\end{itemize}
W oparciu o te zasady można przeprowadzić realistyczną symulację zachowań stadnych.
\section{Pierwszy kamień milowy - stworzenie symulacji}
Pierwszym celem projektu było stworzenie symulacji, w której będziemy mogli uruchamiać algorytm. Zdecydowano się na implementację w języku Python, z wykorzystaniem frameworka pyGame, udostępniającego wiele przydatnych narzędzi do tworzenia prostych animowanych wizualizacji.\\\\
Podstawowa wersja symulacji tworzyła osobniki posiadające pewien zakres widzenia, który został na stałe zaszyty w kodzie programu. W oparciu o ten zakres, agenci mogli dostrzegać sąsiadów w celu łączenia się w grupy. Nie reagowali oni jednak początkowo na przeszkody - uderzenie w przeszkodę kończyło się śmiercią osobnika i usunięciem go z symulacji. Zdecydowaliśmy się jednak na zmianę tego podejścia już na początkowym etapie, ponieważ poprzez szybkie wymieranie agentów nie moglibyśmy obserwować rozwoju działania algorytmu w dłuższej perspektywie. W związku z tym, implementacja została dostosowana - osobniki, które zginęły na wskutek zderzenia z przeszkodą były automatycznie zastępowane nowymi osobnikami, generowanymi w losowej pozycji na ekranie. 
\\\\Podstawowa wersja symulacji obejmowała również możliwość dynamicznego dodawania przeszkód na ekranie poprzez kliknięcie. Niemniej jednak, uznaliśmy że symulacja powinna działać całkowicie samoczynnie, w związku z czym zdecydowaliśmy, że przeszkody powinny być generowane losowo, w sposób deterministyczny przy uruchomieniu symulacji - w taki też sposób dostosowaliśmy program. 

\begin{figure}[H]
    \centering
    \includegraphics[width=1\linewidth]{obraz1.png}
    \caption{Finalna wersja GUI symulacji}
    \label{fig:enter-label}
\end{figure}

\section{Drugi kamień milowy - implementacja podstawowego algorytmu}
Po zaakceptowaniu implementacji samej symulacji, przeszliśmy do wdrożenia algorytmu stadnego. \\\\
W przypadku naszego projektu, agentami były ptaki. Program zaimplementowaliśmy w oparciu o paradygmat programowania obiektowego. Każdy z agentów posiadał zasięg widzenia sąsiadów, oraz zasięg widzenia przeszkód. W oparciu o te funkcjonalności, ptaki mogły z założenia unikać przeszkód oraz trzymać się w stadzie ze swoimi sąsiadami. Nominalny dobór parametrów na poziomie:
\begin{itemize}
    \item współczynnik kohezji - waga 0.02
    \item współczynnik separacji - waga 0.01
    \item współczynnik wyrównania - waga 0.02
\end{itemize}
pozwolił na bezproblemową realizację utrzymania ptaków w stadzie po tym, jak co najmniej dwa z nich znalazły się w zasięgu swojej wizji.\\\\
Sytuacja wygląda podobnie w przypadku pojawienia się przeszkody w zasięgu wizji agenta. Gdy pojawiła się przeszkoda, wektor prędkości aktualizowany był taki, aby agent mógł ominąć przeszkodę. W przypadku samodzielnego lotu, agent całkowicie zmieniał kierunek, natomiast gdy poruszał się w stadzie, starał się ominąć przeszkodę utrzymując się jednocześnie w formacji.
\section{Trzeci kamień milowy - optymalizacja algorytmu}
Ostatnim etapem naszego projektu była optymalizacja algorytmu przez odpowiedni dobór parametrów kohezji, separacji oraz wyrównania. W ramach własnej inwencji dodaliśmy jeszcze jeden parametr, określający jak mocno zmiana wektora prędkości przy zauważeniu przeszkody ma wpływać na zmianę kierunku lotu agenta. \\\\
Tak jak wspomniano wcześniej, dobrane wcześniej wagi przy współczynnikach działały dobrze. Dokonano jednak testów poprzez ich zmianę, co było widoczne w zachowaniu lecących w stadzie ptaków, które zależnie od parametrów trzymały się bliżej lub dalej od siebie, lub ich czas dołączenia do stada i wyrównania kierunku lotu był dłuższy lub krótszy. Zdecydowaliśmy się jednak przyjąć wspomniane w poprzedniej sekcji wielkości parametrów jako odpowiednie. \\\\
Skupiliśmy się jednak na próbie optymalizacji parametru odpowiedzialnego za zmianę kierunku przy napotkaniu przeszkody. O ile w przypadku lotu pojedynczego osobnika przeszkody były bez problemu omijane, o tyle w przypadku lotu w grupie osobniki gorzej radziły sobie z wymijaniem przeszkód. Nawet przy ustawieniu większej wartości wspomnianego współczynnika omijania, agenci mieli tendencję do uderzania w przeszkodę ze względu na priorytetyzowanie zachowania stadnego ponad ominięcie przeszkody. Niestety, w tym przypadku optymalizacja parametru okazała się niewystarczająca - należałoby zaimiplementować dodatkowe mechanizmy, które umożliwiłyby agentom chwilowe przerywanie szyku w celu ominięcia przeszkody. 
\section{Podsumowanie i wnioski}
Podsumowując, udało się z powodzeniem zaimplementować i przetestować empirycznie podstawową wersję algorytmu stadnego. Agenci poprawnie gromadzili się w stada, posiadali zdolność do przyjęcia i utrzymania wspólnego kierunku lotu, oraz byli w stanie formować się w szyk. Niestety, algorytm ten nie uwzględniał omijania przeszkód przez agentów - w naturze, osobniki robią to instynktownie, co oczywiście nie jest odzwierciedlowe w symulacji. W celu poprawy należałoby zamiplementować dodatkowe mechanizmy, pozwalające agentom na ominięcie przeszkód napotykanych przez nich na torze lotu. \\

Repozytorium kodu:
\url{https://github.com/StaniszewskiA/SSIM-Proj}

\end{document}
